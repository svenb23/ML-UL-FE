\documentclass[a4paper,11pt]{article}

\usepackage[utf8]{inputenc}
\usepackage[T1]{fontenc}
\usepackage[ngerman]{babel}
\usepackage{graphicx}
\usepackage{array}
\usepackage{booktabs}
\usepackage{amsmath}
\usepackage{float}

\usepackage[scaled]{helvet}
\renewcommand{\familydefault}{\sfdefault}

\usepackage[a4paper, top=2cm, bottom=2cm, left=2cm, right=2cm]{geometry}

\usepackage{setspace}
\setstretch{1.5}

\setlength{\parindent}{0pt}
\setlength{\parskip}{6pt}

\usepackage{microtype}
\sloppy
\hyphenpenalty=1000
\tolerance=3000

\renewcommand{\footnotesize}{\fontsize{10}{12}\selectfont}

\setcounter{secnumdepth}{3}
\setcounter{tocdepth}{3}

\usepackage{titlesec}
\titleformat{\section}{\normalfont\fontsize{12}{14}\bfseries}{\thesection}{1em}{}
\titleformat{\subsection}{\normalfont\fontsize{12}{14}\bfseries}{\thesubsection}{1em}{}
\titleformat{\subsubsection}{\normalfont\fontsize{12}{14}\bfseries}{\thesubsubsection}{1em}{}

\usepackage[
  colorlinks=true,
  linkcolor=black,
  citecolor=blue,
  filecolor=black,
  urlcolor=blue
]{hyperref}
\usepackage[capitalise,nameinlink]{cleveref}

\usepackage{fancyhdr}
\pagestyle{fancy}
\fancyhf{}
\renewcommand{\headrulewidth}{0pt}
\fancyfoot[C]{\thepage}

\usepackage[backend=biber, style=apa]{biblatex}
\addbibresource{references.bib}

\usepackage{titling}

\usepackage{acronym}

\usepackage{caption}
\usepackage{threeparttable}
\captionsetup[table]{
    font=small,
    skip=10pt,
    labelfont=bf
}

\usepackage{listings}
\usepackage{xcolor}

\definecolor{codegreen}{rgb}{0,0.6,0}
\definecolor{codegray}{rgb}{0.5,0.5,0.5}
\definecolor{codepurple}{rgb}{0.58,0,0.82}
\definecolor{backcolour}{rgb}{0.95,0.95,0.92}

\lstdefinestyle{mystyle}{
    backgroundcolor=\color{backcolour},
    commentstyle=\color{codegreen},
    keywordstyle=\color{magenta},
    numberstyle=\tiny\color{codegray},
    stringstyle=\color{codepurple},
    basicstyle=\ttfamily\footnotesize,
    breakatwhitespace=false,
    breaklines=true,
    captionpos=b,
    keepspaces=true,
    numbers=left,
    numbersep=5pt,
    showspaces=false,
    showstringspaces=false,
    showtabs=false,
    tabsize=2
}
\lstset{style=mystyle}

\begin{document}

\begin{titlepage}
    \thispagestyle{empty}
    \centering
    \vspace*{5cm}
    {\Huge\bfseries Projekt: 
Maschinelles Lernen - Unsupervised Learning und Feature Engineering DLBDSMLUSL01\_D \par}
    \vspace{1cm}
    {\Large Fallstudie \par}
    \vspace{0.5cm}
    {\large Studiengang: Angewandte Künstliche Intelligenz \par}
    \vspace{0.5cm}
    {\large Sven Behrens \par}
    \vspace{0.5cm}
    {\large Matrikelnummer: 42303511 \par}
    \vspace{0.5cm}
    {\large Prof. Dr. Christian Müller-Kett \par}
    \vspace{0.5cm}
    {\large \today \par}
\end{titlepage}

\pagenumbering{Roman}
\setcounter{page}{1}

\tableofcontents
\newpage

\listoffigures
\addcontentsline{toc}{section}{Abbildungsverzeichnis}
\newpage

\listoftables
\addcontentsline{toc}{section}{Tabellenverzeichnis}
\newpage

\section*{Abkürzungsverzeichnis}
\addcontentsline{toc}{section}{Abkürzungsverzeichnis}
\begin{acronym}[t-SNE]
    \acro{AFAB}{Assigned Female At Birth}
    \acro{GMM}{Gaussian Mixture Models}
    \acro{LLE}{Locally Linear Embedding}
    \acro{MDS}{Multidimensional Scaling}
    \acro{OSMI}{Open Sourcing Mental Illness}
    \acro{PCA}{Principal Component Analysis}
    \acro{t-SNE}{t-Distributed Stochastic Neighbor Embedding}
    \acro{UMAP}{Uniform Manifold Approximation and Projection}
\end{acronym}
\newpage

\pagenumbering{arabic}
\setcounter{page}{1}

\section{Einleitung}
Immer mehr Menschen erkranken an psychischen Erkrankungen \parencite{who2025mentalhealth}. Auch in technologieorientierten Berufen
rücken psychische Belastungen am Arbeitsplatz verstärkt in den Fokus von Unternehmen und Forschung. Die systematische Analyse von Umfragedaten 
zur psychischen Gesundheit stellt dabei eine zentrale Herausforderung dar, deren Bewältigung maßgeblich zur Entwicklung 
zielgerichteter Präventionsprogramme und zur Verbesserung der Arbeitsbedingungen beitragen kann. Vor diesem Hintergrund 
wurde im Rahmen des Moduls „Projekt: Maschinelles Lernen -- Unsupervised Learning und Feature Engineering" an der
IU Internationalen Hochschule eine umfassende Clusteranalyse von Umfragedaten zur psychischen Gesundheit in der 
Technologiebranche durchgeführt.

Das primäre Projektziel bestand in der Kategorisierung von Umfrageteilnehmenden anhand ihrer Antworten zu psychischen 
Belastungen, Arbeitgeberunterstützung und Stigmatisierungserfahrungen mittels unüberwachter Lernverfahren. 
Die zentrale Forschungsfrage konzentrierte sich darauf, wie durch den Einsatz verschiedener Dimensionsreduktions- und Clustering-Methoden 
aussagekräftige Teilnehmergruppen identifiziert werden können, die als Grundlage für gezielte Interventionsmaßnahmen der Personalabteilung 
dienen. Besondere Aufmerksamkeit galt dabei der Interpretierbarkeit der Ergebnisse sowie der Reduktion der hohen Dimensionalität des 
Datensatzes bei gleichzeitiger Beibehaltung der wesentlichen Informationsstruktur.

Die Datenbasis bildete der OSMI Mental Health in Tech Survey 2016 \parencite{osmi2016mentalhealth}, der auf Kaggle frei verfügbar ist und Antworten 
von 1.433 Beschäftigten aus technologieorientierten Unternehmen umfasst. Der Datensatz enthält 63 Fragen zu Themen 
wie diagnostizierte psychische Erkrankungen, Einstellungen gegenüber psychischer Gesundheit am Arbeitsplatz, 
wahrgenommene Arbeitgeberunterstützung, Stigmatisierungserfahrungen sowie demografische Merkmale. Die Herausforderungen 
bei der Arbeit mit diesen Daten lagen insbesondere in der hohen Anzahl fehlender Werte, nicht standardisierten 
Texteingaben z.B bei Geschlechtsangaben, sowie der Notwendigkeit einer geeigneten Kodierung kategorischer Variablen für maschinelle Lernalgorithmen.

Die methodische Vorgehensweise gliederte sich in mehrere aufeinander aufbauende Phasen. Nach einer initialen 
explorativen Datenanalyse erfolgte eine umfassende Datenvorverarbeitung, die Bereinigung, Normalisierung, 
Imputation fehlender Werte sowie Feature Engineering umfasste. Anschließend wurden verschiedene Dimensionsreduktionsmethoden, darunter
\ac{PCA}, \ac{t-SNE}, \ac{UMAP}, \ac{MDS} und
\ac{LLE}, systematisch evaluiert und verglichen. Für die Clusteranalyse kamen K-Means, \ac{GMM}
sowie hierarchisches Clustering zum Einsatz, wobei die optimale Clusteranzahl durch verschiedene Evaluationsmetriken bestimmt wurde.

Der gewählte Ansatz zeichnet sich durch seine iterative Vorgehensweise aus, bei der in drei aufeinander aufbauenden Durchläufen 
die Datenvorverarbeitung und Modellierung schrittweise verfeinert wurden. Durch die Kombination verschiedener Methoden und die sorgfältige 
Interpretation der Ergebnisse wurde ein Analysesystem geschaffen, das nicht nur statistisch fundierte Cluster identifiziert, sondern auch 
praxisrelevante Erkenntnisse für die Gestaltung von Präventionsprogrammen liefert.

Die vorliegende Fallstudie dokumentiert systematisch den gesamten Analyseprozess. Nach dieser Einleitung folgt
die detaillierte Beschreibung der Datenbasis sowie der durchgeführten Vorverarbeitungsschritte. Anschließend werden die
angewandten Dimensionsreduktions- und Clustering-Methoden erläutert und deren Ergebnisse ausführlich dargestellt.
Abschließend werden die gewonnenen Erkenntnisse in einem Fazit zusammengefasst und kritisch reflektiert. Da die Analyse
sensibler Gesundheitsdaten besondere Sorgfalt erfordert, werden zudem ethische und gesellschaftliche Aspekte wie
Datenschutz, Stigmatisierungsrisiken und der verantwortungsvolle Umgang mit den Ergebnissen eingehend diskutiert.

\section{Hauptteil}

\subsection{Projektumgebung}
Zu Beginn des Projekts wurde ein GitHub-Repository\footnote{\url{https://github.com/svenb23/ML-UL-FE}} angelegt, um eine nachvollziehbare Versionsverwaltung zu gewährleisten
und bei Bedarf auf frühere Entwicklungsstände zurückgreifen zu können. Anschließend wurde eine grundlegende
Verzeichnisstruktur erstellt, die separate Ordner für Notebooks, Daten, Visualisierungen und den Bericht umfasst.
Für die Entwicklung wurde mit venv eine virtuelle Python-Umgebung eingerichtet, in der die benötigten
Bibliotheken wie NumPy, Pandas, Scikit-learn, Matplotlib, Seaborn und UMAP-learn installiert wurden. Als
Entwicklungsumgebung dienten Jupyter Notebooks. Abschließend wurde der OSMI Mental Health in Tech Survey 2016
Datensatz von Kaggle heruntergeladen und im Datenverzeichnis abgelegt.

\subsection{Iterativer Analyseansatz}
Die Analyse wurde in einem iterativen Prozess durchgeführt, der drei aufeinander aufbauende Durchläufe umfasste.
Dieser Ansatz ermöglichte es, aus den Erkenntnissen jedes Durchlaufs systematisch zu lernen und die Methodik
kontinuierlich zu verbessern. Jeder Durchlauf folgte dabei einem einheitlichen Ablauf bestehend aus
Datenvorverarbeitung, Dimensionsreduktion, Clustering und anschließender Interpretation der Ergebnisse.

\subsection{Erster Durchlauf}
\subsubsection{Datenexploration und Vorverarbeitung}
Der Datensatz umfasste 1.433 Teilnehmende und 63 Features, die verschiedene Aspekte der psychischen Gesundheit
am Arbeitsplatz abdeckten. Die Features ließen sich thematisch in demografische Merkmale, Angaben zum
Arbeitsumfeld, persönliche psychische Gesundheitssituation, Unterstützung durch aktuelle und frühere Arbeitgeber
sowie Einstellungen und Stigmatisierungserfahrungen einteilen. Eine initiale Prüfung auf offensichtlich
irrelevante Spalten ergab keine Kandidaten und wurde aufgrund mangelnden domänenspezifischen
Wissens nicht weiter verfolgt.

Die Analyse der Datentypen zeigte eine deutliche Dominanz kategorischer Variablen: 56 der 63 Features lagen als
Objekttyp vor, während lediglich vier ganzzahlige und drei Gleitkomma-Spalten existierten. Eine differenziertere
Betrachtung der Datentypen ergab die in \cref{tab:datentypen_run01} dargestellte Verteilung.

Bei der Ausreißeranalyse wurde die Altersspalte untersucht, da diese als einzige numerische Variable anfällig
für Fehleingaben war. Dabei wurden sechs auffällige Werte identifiziert: 3, 15, 17, 74, 99 und 323 Jahre.
Die Werte 3, 99 und 323 wurden als offensichtliche Fehleingaben klassifiziert und aus dem Datensatz entfernt.
Die Grenzfälle 15, 17 und 74 Jahre wurden beibehalten, da sie im Kontext der Technologiebranche als möglich
eingestuft wurden.

Das Freitextfeld für Geschlechtsangaben führte zu 69 unterschiedlichen Einträgen. Diese reichten von
Schreibvarianten wie „Male", „male", „M" und „Man" über Bezeichnungen wie „non-binary" und „Agender"
bis hin zu „Transgender woman" oder „AFAB". Da eine differenzierte Betrachtung von biologischem und
empfundenem Geschlecht über den Fokus dieser Analyse hinausgeht, wurden die Angaben pragmatisch auf Basis
des bei Geburt zugewiesenen Geschlechts in drei Kategorien zusammengefasst: Male (1.055), Female (338)
und Other (34).

Die Analyse fehlender Werte ergab, dass 44 der 63 Spalten Lücken aufwiesen (siehe \cref{tab:missing_values_run01}).
Der Anteil reichte von 0,2\% bis zu 89,9\% pro Spalte. Spalten mit mehr als 70\% fehlenden Werten wurden entfernt, da eine sinnvolle
Imputation bei diesem Anteil nicht mehr gewährleistet werden konnte. Ebenso wurden die US-Bundesstaaten-Spalten
ausgeschlossen, da sie nur für amerikanische Teilnehmende relevant waren. Für Fragen zum aktuellen Arbeitgeber
wurden fehlende Werte bei Selbstständigen mit „Not applicable" gefüllt, da diese Fragen für sie nicht zutrafen.
Die verbleibenden fehlenden Werte in kategorischen Spalten wurden mit dem Modus imputiert.

\subsubsection{Dimensionsreduktion}
% TODO: PCA, t-SNE, UMAP, MDS, LLE

\subsubsection{Clustering}
% TODO: k-Means, GMM, Hierarchisches Clustering

\subsubsection{Erkenntnisse}
% TODO: Was wurde gelernt? Was muss verbessert werden?

\subsection{Zweiter Durchlauf}
% TODO: run_02

\subsubsection{Anpassungen der Vorverarbeitung}
% TODO: Verbesserungen basierend auf run_01

\subsubsection{Dimensionsreduktion}
% TODO: Angepasste Analyse

\subsubsection{Clustering}
% TODO: Angepasste Analyse

\subsubsection{Erkenntnisse}
% TODO: Was wurde gelernt? Was muss verbessert werden?

\subsection{Dritter Durchlauf}
% TODO: run_03

\subsubsection{Finale Vorverarbeitung}
% TODO: Composite Indices, Feature Selection, KNN-Imputation

\subsubsection{Dimensionsreduktion}
% TODO: Finale Analyse mit PCA Loadings

\subsubsection{Clustering}
% TODO: Finale Clusteranalyse mit k=4

\subsubsection{Ergebnisse und Interpretation}
% TODO: Cluster-Charakterisierung, Handlungsempfehlungen


\section{Fazit}

\subsection{Zielerreichung und Projektergebnisse}
% TODO: Zusammenfassung der Zielerreichung

\subsection{Kritische Reflexion}
% TODO: Kritische Reflexion

\subsection{Ethische und gesellschaftliche Aspekte}
% TODO: Ethische Betrachtungen

\subsection{Ausblick}
% TODO: Ausblick und Verbesserungspotenziale


\newpage

\printbibliography
\addcontentsline{toc}{section}{Literaturverzeichnis}

\newpage
\section*{Verzeichnis der Anhänge}
\addcontentsline{toc}{section}{Verzeichnis der Anhänge}

\appendix
\section*{Anhang}
\addcontentsline{toc}{section}{Anhang}

\begin{table}[H]
\centering
\caption{Verteilung der Datentypen nach Skalenniveau}
\label{tab:datentypen_run01}
\begin{tabular}{llr}
\toprule
Haupttyp & Untertyp & Anzahl \\
\midrule
Kategorisch & Nominal & 42 \\
Kategorisch & Ordinal & 7 \\
Kategorisch & Binär & 7 \\
Kategorisch & Multi-Value & 4 \\
Text & Freitext & 2 \\
Numerisch & Verhältnis & 1 \\
\bottomrule
\end{tabular}
\end{table}

\begin{table}[H]
\centering
\caption{Spalten mit fehlenden Werten (Auszug)}
\label{tab:missing_values_run01}
\begin{tabular}{rrl}
\toprule
Anzahl & Anteil & Spalte \\
\midrule
1286 & 89,9\% & Revealed to client impacted negatively \\
1226 & 85,7\% & Work time affected percentage \\
1167 & 81,6\% & Primary role related to tech/IT \\
1143 & 79,9\% & Know local resources for help \\
1143 & 79,9\% & Reveal diagnosis to clients \\
1143 & 79,9\% & Revealed to coworker impacted negatively \\
1143 & 79,9\% & Productivity affected by mental health \\
1109 & 77,6\% & Conditions believed to have \\
863 & 60,3\% & Diagnosed conditions \\
775 & 54,2\% & Observations made less likely to reveal \\
721 & 50,4\% & Professional diagnosed conditions \\
592 & 41,4\% & US state (live) \\
581 & 40,6\% & US state (work) \\
420 & 29,4\% & Know employer coverage options \\
287 & 20,1\% & Comfortable discussing with coworkers \\
\bottomrule
\end{tabular}
\end{table}

\end{document}
